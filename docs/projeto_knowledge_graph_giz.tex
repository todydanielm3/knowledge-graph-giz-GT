\documentclass[12pt,a4paper]{article}
\usepackage[utf8]{inputenc}
\usepackage[portuguese]{babel}
\usepackage[T1]{fontenc}
\usepackage{geometry}
\usepackage{graphicx}
\usepackage{hyperref}
\usepackage{listings}
\usepackage{xcolor}
\usepackage{fancyhdr}
\usepackage{titlesec}
\usepackage{enumitem}
\usepackage{amsmath}
\usepackage{booktabs}
\usepackage{multicol}
\usepackage{tikz}
\usepackage{pgfplots}

% Configurações de página
\geometry{margin=2.5cm}
\pagestyle{fancy}
\fancyhf{}
\fancyhead[L]{\textbf{Knowledge Graph GIZ-BR}}
\fancyhead[R]{\thepage}
\fancyfoot[C]{\footnotesize Projeto de Grafo de Conhecimento - GIZ Brasil}

% Configurações de código
\definecolor{codegreen}{rgb}{0,0.6,0}
\definecolor{codegray}{rgb}{0.5,0.5,0.5}
\definecolor{codepurple}{rgb}{0.58,0,0.82}
\definecolor{backcolour}{rgb}{0.95,0.95,0.92}

\lstdefinestyle{mystyle}{
    backgroundcolor=\color{backcolour},   
    commentstyle=\color{codegreen},
    keywordstyle=\color{magenta},
    numberstyle=\tiny\color{codegray},
    stringstyle=\color{codepurple},
    basicstyle=\ttfamily\footnotesize,
    breakatwhitespace=false,         
    breaklines=true,                 
    captionpos=b,                    
    keepspaces=true,                 
    numbers=left,                    
    numbersep=5pt,                  
    showspaces=false,                
    showstringspaces=false,
    showtabs=false,                  
    tabsize=2
}

\lstset{style=mystyle}

% Configurações de títulos
\titleformat{\section}
{\normalfont\Large\bfseries\color{blue!70!black}}{\thesection}{1em}{}

\titleformat{\subsection}
{\normalfont\large\bfseries\color{blue!50!black}}{\thesubsection}{1em}{}

% Configurações de hyperlinks
\hypersetup{
    colorlinks=true,
    linkcolor=blue,
    filecolor=magenta,      
    urlcolor=cyan,
    pdftitle={Knowledge Graph GIZ-BR},
    pdfauthor={GIZ Brasil},
}

\begin{document}

% Página de título
\begin{titlepage}
    \centering
    \vspace*{2cm}
    
    {\huge\bfseries Knowledge Graph para Produtos Digitais}\\[0.5cm]
    {\Large\bfseries GIZ-Brasil}\\[2cm]
    
    {\large Protótipo de Sistema de Grafo de Conhecimento}\\
    {\large para Gestão de Projetos e Metadados}\\[3cm]
    
    \begin{tikzpicture}[scale=0.8]
        % Nós do grafo
        \node[circle, draw, fill=red!20, minimum size=1.5cm] (p1) at (0,0) {\small Projeto 1};
        \node[circle, draw, fill=green!20, minimum size=1.2cm] (o1) at (3,1) {\small GIZ};
        \node[circle, draw, fill=blue!20, minimum size=1.2cm] (t1) at (3,-1) {\small Tema};
        \node[circle, draw, fill=orange!20, minimum size=1.2cm] (p2) at (6,0) {\small Projeto 2};
        
        % Arestas
        \draw[->] (p1) -- (o1) node[midway, above] {\tiny executado\_por};
        \draw[->] (p1) -- (t1) node[midway, below] {\tiny aborda};
        \draw[->] (p1) -- (p2) node[midway, above] {\tiny relacionado\_com};
        \draw[->] (p2) -- (o1) node[midway, above right] {\tiny executado\_por};
    \end{tikzpicture}\\[2cm]
    
    {\large\today}\\[1cm]
    
    {\normalsize
    \textbf{Tecnologias:} Python, Neo4j, Streamlit, Docker\\
    \textbf{Versão:} 1.0\\
    \textbf{Licença:} MIT
    }
\end{titlepage}

\tableofcontents
\newpage

\section{Resumo Executivo}

O projeto \textbf{Knowledge Graph GIZ-BR} é um protótipo completo que consolida metadados de projetos da GIZ-Brasil em um banco de dados orientado a grafos (Neo4j) e oferece uma interface web interativa através do Streamlit para visualização e análise das relações entre projetos, organizações e temas.

\subsection{Objetivos Principais}
\begin{itemize}
    \item Centralizar informações de projetos em um formato estruturado
    \item Identificar relações e dependências entre projetos
    \item Facilitar a descoberta de conhecimento através de visualizações interativas
    \item Prover uma plataforma escalável para análise de portfólio de projetos
\end{itemize}

\subsection{Benefícios}
\begin{itemize}
    \item \textbf{Visibilidade:} Visualização clara das conexões entre projetos
    \item \textbf{Análise:} Identificação de padrões e oportunidades de sinergia
    \item \textbf{Gestão:} Melhor tomada de decisão baseada em dados relacionais
    \item \textbf{Escalabilidade:} Arquitetura preparada para crescimento
\end{itemize}

\section{Arquitetura do Sistema}

\subsection{Visão Geral}

O sistema é composto por três camadas principais:

\begin{enumerate}
    \item \textbf{Camada de Dados}: Neo4j como banco de grafos
    \item \textbf{Camada de Processamento}: Scripts Python para ETL
    \item \textbf{Camada de Apresentação}: Interface web Streamlit
\end{enumerate}

\subsection{Tecnologias Utilizadas}

\begin{table}[h]
\centering
\begin{tabular}{@{}lll@{}}
\toprule
\textbf{Componente} & \textbf{Tecnologia} & \textbf{Versão} \\
\midrule
Banco de Dados & Neo4j Community & 5.20+ \\
Backend & Python & 3.10+ \\
Frontend & Streamlit & 1.35+ \\
Visualização & PyVis & 0.3.1+ \\
Containerização & Docker & 20.10+ \\
Processamento de Dados & Pandas & 2.2+ \\
\bottomrule
\end{tabular}
\caption{Stack Tecnológica do Projeto}
\end{table}

\subsection{Arquitetura de Dados}

O modelo de dados é baseado em grafos, onde:

\begin{itemize}
    \item \textbf{Nós (Vértices):} Representam entidades como Projetos, Organizações e Temas
    \item \textbf{Arestas (Relacionamentos):} Representam conexões como "executado\_por", "aborda", "relacionado\_com"
    \item \textbf{Propriedades:} Metadados associados aos nós (nome, status, orçamento, etc.)
\end{itemize}

\section{Componentes do Sistema}

\subsection{ETL (Extract, Transform, Load)}

O módulo ETL é responsável pela ingestão de dados:

\begin{lstlisting}[language=Python, caption=Função de Upsert de Projetos]
def upsert_project(tx, props: dict):
    """
    Garante um nó Project com id único e
    atualiza/insere todas as demais propriedades.
    """
    tx.run(
        """
        MERGE (p:Project {id: $id})
        SET   p += $props
        """,
        id=props["id"],
        props=props
    )
\end{lstlisting}

\textbf{Características:}
\begin{itemize}
    \item Operações MERGE para evitar duplicatas
    \item Processamento incremental de dados
    \item Tratamento de diferentes tipos de dados
    \item Log de operações realizadas
\end{itemize}

\subsection{Interface Web (Streamlit)}

A aplicação web oferece:

\begin{itemize}
    \item \textbf{Seleção de Projetos:} Dropdown com todos os projetos disponíveis
    \item \textbf{Informações Detalhadas:} Exibição de metadados do projeto selecionado
    \item \textbf{Visualização de Grafo:} Representação interativa usando PyVis
    \item \textbf{Estatísticas:} Métricas sobre conectividade e relações
    \item \textbf{Dados Tabulares:} Fallback em formato tabela para análise detalhada
\end{itemize}

\subsection{Banco de Dados Neo4j}

Configuração do banco:

\begin{lstlisting}[language=yaml, caption=Docker Compose para Neo4j]
services:
  neo4j:
    image: neo4j:5.20-community
    restart: unless-stopped
    ports:
      - "7474:7474"  # Interface web
      - "7687:7687"  # Protocolo Bolt
    environment:
      - NEO4J_AUTH=neo4j/test12345
      - NEO4J_PLUGINS=["apoc"]
\end{lstlisting}

\section{Modelo de Dados}

\subsection{Entidades Principais}

\begin{enumerate}
    \item \textbf{Project}
    \begin{itemize}
        \item Propriedades: id, name, status, start\_date, budget
        \item Exemplo: AdaptaInfra, Gênero\&Infra, Clima-Amazônia
    \end{itemize}
    
    \item \textbf{Organizacao}
    \begin{itemize}
        \item Propriedades: name, tipo
        \item Exemplo: GIZ Brasil, Ministério da Infraestrutura
    \end{itemize}
    
    \item \textbf{Tema}
    \begin{itemize}
        \item Propriedades: name, area
        \item Exemplo: Sustentabilidade, Igualdade de Gênero
    \end{itemize}
\end{enumerate}

\subsection{Relacionamentos}

\begin{table}[h]
\centering
\begin{tabular}{@{}lll@{}}
\toprule
\textbf{Origem} & \textbf{Relacionamento} & \textbf{Destino} \\
\midrule
Project & EXECUTADO\_POR & Organizacao \\
Project & ABORDA & Tema \\
Project & RELACIONADO\_COM & Project \\
Project & INFLUENCIA & Project \\
\bottomrule
\end{tabular}
\caption{Tipos de Relacionamentos no Grafo}
\end{table}

\section{Funcionalidades}

\subsection{Visualização Interativa}

A interface oferece:

\begin{itemize}
    \item \textbf{Grafo Dinâmico:} Visualização em tempo real das conexões
    \item \textbf{Nós Coloridos:} Diferenciação visual por tipo de entidade
    \item \textbf{Tooltips Informativos:} Detalhes ao passar o mouse
    \item \textbf{Layout Adaptativo:} Organização automática dos elementos
\end{itemize}

\subsection{Análise de Dados}

\begin{itemize}
    \item \textbf{Métricas de Conectividade:} Número de relações por projeto
    \item \textbf{Estatísticas do Grafo:} Contadores de entidades conectadas
    \item \textbf{Dados Estruturados:} Exportação em formato tabular
    \item \textbf{Filtros:} Seleção específica por projeto
\end{itemize}

\section{Guia de Instalação}

\subsection{Pré-requisitos}

\begin{itemize}
    \item Python 3.10 ou superior
    \item Docker e Docker Compose
    \item Git (para clonagem do repositório)
    \item 4GB RAM disponível (para Neo4j)
\end{itemize}

\subsection{Processo de Instalação}

\begin{enumerate}
    \item \textbf{Clonar o repositório:}
    \begin{lstlisting}[language=bash]
git clone <repository-url>
cd knowledge-graph-giz-GT
    \end{lstlisting}
    
    \item \textbf{Configurar ambiente Python:}
    \begin{lstlisting}[language=bash]
python3 -m venv venv
source venv/bin/activate
pip install -r requirements.txt
    \end{lstlisting}
    
    \item \textbf{Iniciar Neo4j:}
    \begin{lstlisting}[language=bash]
docker compose up -d
    \end{lstlisting}
    
    \item \textbf{Carregar dados:}
    \begin{lstlisting}[language=bash]
python etl/load_metadata.py
    \end{lstlisting}
    
    \item \textbf{Executar aplicação:}
    \begin{lstlisting}[language=bash]
streamlit run app/streamlit_app.py
    \end{lstlisting}
\end{enumerate}

\section{Estrutura do Projeto}

\begin{lstlisting}[language=bash, caption=Organização de Diretórios]
knowledge-graph-giz-GT/
├── app/
│   ├── streamlit_app.py      # Interface principal
│   └── secrets.toml          # Credenciais
├── data/
│   └── projects.csv          # Dados de entrada
├── etl/
│   └── load_metadata.py      # Script ETL
├── docker-compose.yml        # Configuração Neo4j
├── requirements.txt          # Dependências Python
└── README.md                 # Documentação
\end{lstlisting}

\section{Casos de Uso}

\subsection{Gestão de Portfólio}

\begin{itemize}
    \item Identificar projetos relacionados para possível consolidação
    \item Analisar distribuição de recursos por tema
    \item Detectar gaps ou sobreposições no portfólio
\end{itemize}

\subsection{Análise de Impacto}

\begin{itemize}
    \item Avaliar o alcance de temas específicos
    \item Identificar organizações parceiras estratégicas
    \item Mapear dependências entre projetos
\end{itemize}

\subsection{Planejamento Estratégico}

\begin{itemize}
    \item Visualizar a rede de relacionamentos organizacionais
    \item Identificar oportunidades de colaboração
    \item Otimizar alocação de recursos
\end{itemize}

\section{Extensões Futuras}

\subsection{Roadmap Técnico}

\begin{enumerate}
    \item \textbf{Busca Semântica:} Implementação de índices vetoriais
    \item \textbf{APIs REST:} Exposição de endpoints para integração
    \item \textbf{Autenticação:} Sistema de login e controle de acesso
    \item \textbf{Dashboard Analytics:} KPIs e métricas avançadas
    \item \textbf{Machine Learning:} Recomendações e predições
\end{enumerate}

\subsection{Melhorias de Interface}

\begin{enumerate}
    \item Interface responsiva para dispositivos móveis
    \item Filtros avançados por múltiplos critérios
    \item Exportação de visualizações em diferentes formatos
    \item Histórico de navegação e favoritos
    \item Compartilhamento de visualizações via URL
\end{enumerate}

\section{Segurança e Boas Práticas}

\subsection{Segurança}

\begin{itemize}
    \item \textbf{Credenciais:} Armazenamento em arquivos de configuração separados
    \item \textbf{Validação:} Sanitização de entradas de usuário
    \item \textbf{Criptografia:} Comunicação segura via HTTPS (produção)
    \item \textbf{Backup:} Procedimentos de backup do banco de dados
\end{itemize}

\subsection{Performance}

\begin{itemize}
    \item \textbf{Cache:} Uso de cache do Streamlit para consultas
    \item \textbf{Índices:} Criação de índices apropriados no Neo4j
    \item \textbf{Paginação:} Limitação de resultados em consultas grandes
    \item \textbf{Otimização:} Consultas Cypher eficientes
\end{itemize}

\section{Conclusão}

O projeto Knowledge Graph GIZ-BR representa uma solução moderna e escalável para gestão de conhecimento organizacional. Através da combinação de tecnologias de grafos, processamento de dados e visualização interativa, oferece uma plataforma robusta para análise e descoberta de relações em portfólios de projetos.

\subsection{Principais Contribuições}

\begin{itemize}
    \item Arquitetura modular e extensível
    \item Interface intuitiva para usuários não-técnicos
    \item Modelo de dados flexível para diferentes tipos de entidades
    \item Demonstração prática de tecnologias de grafo em contexto organizacional
\end{itemize}

\subsection{Impacto Esperado}

A implementação desta solução pode resultar em:

\begin{itemize}
    \item Melhoria na tomada de decisões estratégicas
    \item Identificação de oportunidades de sinergia
    \item Otimização na alocação de recursos
    \item Maior visibilidade das atividades organizacionais
\end{itemize}

O projeto estabelece uma base sólida para futuras expansões e pode servir como modelo para outras organizações interessadas em implementar soluções similares de gestão de conhecimento.

\newpage
\section{Apêndices}

\subsection{Apêndice A: Comandos Cypher Úteis}

\begin{lstlisting}[language=sql, caption=Consultas Cypher Essenciais]
-- Listar todos os projetos
MATCH (p:Project) RETURN p

-- Encontrar projetos relacionados
MATCH (p1:Project)-[r]-(p2:Project) 
RETURN p1.name, type(r), p2.name

-- Estatísticas do grafo
MATCH (n) RETURN labels(n), count(n)

-- Projetos por organização
MATCH (p:Project)-[:EXECUTADO_POR]->(o:Organizacao)
RETURN o.name, collect(p.name)
\end{lstlisting}

\subsection{Apêndice B: Configurações Avançadas}

\begin{lstlisting}[language=python, caption=Configurações do PyVis]
g.set_options("""
var options = {
  "physics": {
    "enabled": true,
    "stabilization": {"iterations": 100}
  },
  "nodes": {
    "font": {"size": 16}
  },
  "edges": {
    "font": {"size": 14}
  }
}
""")
\end{lstlisting}

\end{document}
